\documentclass[letterpaper]{article}
\usepackage[utf8]{inputenc}
\usepackage[english]{babel}
\usepackage[toc,page]{appendix}
\usepackage{fullpage}
\renewcommand{\familydefault}{\sfdefault}
\addto{\captionsenglish}{\renewcommand{\chaptername}{}}
\title{Flagship Tasks --- Vision}
\author{ThoughtBeam\\Michael DiTore, Haris Khan, Michael Staib}
\date{\today}
\begin{document}
	\maketitle
	\section{Summary}
		Flagship Tasks is a system for managing tasks, the small steps along the road to the completion of any project.

		Developed by web programmers, it has applications that reach beyond software development, into the worlds of corporate projects, small governments, and personal interests. Because it's so simple to use, its users won't need to be software developers to get things done with it. Flagship Tasks is designed with these other needs in mind.

		Flagship Tasks is a web application, highly portable and extremely flexible. Not only can users have easy, instant access to their projects' status from home and work alike, but Flagship Tasks's plugin system and web API adds flexibility that can make it as powerful or as simple as the user's needs demand.
	\section{Features}
		Flagship Tasks presents many desired features of task management systems in a simple, easy-to-use interface.

		The system has the ability to track multiple separate projects with disparate members and lists of tasks. Its users can be signed up for one project or many - and for those who only wanted to leave a comment, Flagship Tasks supports a simple API that allows users to easily submit feedback to a project without being required to register or even leave the project's site. 

		Access control is important. Project leaders can be given special control over a project by becoming its owners, and read or write access to a project or even a particular task can be easily controlled with group-based access mechanisms. System administration can performed by users marked as administrators, if additional powers are needed to resolve an issue.
	\section{Business Case}
	\section{Resources}
	\section{Project Plan}
	\section{Use Cases}
	\section{Glossary}
\end{document}
